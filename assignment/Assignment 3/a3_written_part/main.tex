\documentclass[answers]{exam}
\newif\ifanswers
\answerstrue % comment out to hide answers

\usepackage{lastpage} % Required to determine the last page for the footer
\usepackage{extramarks} % Required for headers and footers
\usepackage[usenames,dvipsnames]{color} % Required for custom colors
\usepackage{graphicx} % Required to insert images
\usepackage{listings} % Required for insertion of code
\usepackage{courier} % Required for the courier font
\usepackage{lipsum} % Used for inserting dummy 'Lorem ipsum' text into the template
\usepackage{enumerate}
\usepackage{subfigure}
\usepackage{booktabs}
\usepackage{amsmath, amsthm, amssymb}
\usepackage{hyperref}
\usepackage{datetime}
\usepackage{minted}
\settimeformat{ampmtime}
\usepackage{algpseudocode}
\usepackage{algorithmicx}
\usepackage[ruled]{algorithm}
\usepackage{tikz-dependency}

\usepackage{tikz}
\usepackage{amsfonts}
\usetikzlibrary{positioning,patterns,fit,calc}
% Margins
\topmargin=-0.45in
\evensidemargin=0in
\oddsidemargin=0in
\textwidth=6.5in
\textheight=9.0in
\headsep=0.25in

\linespread{1.1} % Line spacing

% Set up the header and footer
%\pagestyle{fancy}
%\rhead{\hmwkAuthorName} % Top left header
%\lhead{\hmwkClass: \hmwkTitle} % Top center head
%\lfoot{\lastxmark} % Bottom left footer
%\cfoot{} % Bottom center footer
%\rfoot{Page\ \thepage\ of\ \protect\pageref{LastPage}} % Bottom right footer
%\renewcommand\headrulewidth{0.4pt} % Size of the header rule
%\renewcommand\footrulewidth{0.4pt} % Size of the footer rule

\pagestyle{headandfoot}
\runningheadrule{}
\firstpageheader{CS 224n}{Assignment 3}{}
\runningheader{CS 224n} {Assignment 3} {Page \thepage\ of \numpages}
\firstpagefooter{}{}{} \runningfooter{}{}{}

\setlength\parindent{0pt} % Removes all indentation from paragraphs

%----------------------------------------------------------------------------------------
%	CODE INCLUSION CONFIGURATION
%----------------------------------------------------------------------------------------

\definecolor{MyDarkGreen}{rgb}{0.0,0.4,0.0} % This is the color used for comments
\lstloadlanguages{Python} % Load Perl syntax for listings, for a list of other languages supported see: ftp://ftp.tex.ac.uk/tex-archive/macros/latex/contrib/listings/listings.pdf
\lstset{language=Python, % Use Perl in this example
    frame=single, % Single frame around code
    basicstyle=\footnotesize\ttfamily, % Use small true type font
    keywordstyle=[1]\color{Blue}\bf, % Perl functions bold and blue
    keywordstyle=[2]\color{Purple}, % Perl function arguments purple
    keywordstyle=[3]\color{Blue}\underbar, % Custom functions underlined and blue
    identifierstyle=, % Nothing special about identifiers
    commentstyle=\usefont{T1}{pcr}{m}{sl}\color{MyDarkGreen}\small, % Comments small dark green courier font
    stringstyle=\color{Purple}, % Strings are purple
    showstringspaces=false, % Don't put marks in string spaces
    tabsize=5, % 5 spaces per tab
%
% Put standard Perl functions not included in the default language here
    morekeywords={rand},
%
% Put Perl function parameters here
    morekeywords=[2]{on, off, interp},
%
% Put user defined functions here
    morekeywords=[3]{test},
%
    morecomment=[l][\color{Blue}]{...}, % Line continuation (...) like blue comment
    numbers=left, % Line numbers on left
    firstnumber=1, % Line numbers start with line 1
    numberstyle=\tiny\color{Blue}, % Line numbers are blue and small
    stepnumber=5 % Line numbers go in steps of 5
}

%----------------------------------------------------------------------------------------
%	NAME AND CLASS SECTION
%----------------------------------------------------------------------------------------

\newcommand{\hmwkTitle}{Dependency Parsing} % Assignment title
\newcommand{\hmwkClass}{CS\ 224n Assignment \#3} % Course/class
\newcommand{\ifans}[1]{\ifanswers \color{red} \textbf{Solution: } #1 \color{black} \fi}

\input macros.tex
\input std_macros.tex

%----------------------------------------------------------------------------------------
%	TITLE PAGE
%----------------------------------------------------------------------------------------
\qformat{\Large\bfseries\thequestion{}. \thequestiontitle{} (\thepoints{})\hfill}

\title{
    \vspace{-1in}
    \textmd{\textbf{\hmwkClass:\ \hmwkTitle} \\ \hmwkAuthorName}\\
}
\author{}
%\date{\textit{\small Updated \today\ at \currenttime}} % Insert date here if you want it to appear below your name
\date{}

\setcounter{section}{0} % one-indexing
\begin{document}

    \maketitle

    \begin{center}
        \large{\textbf{Due on} Tuesday Jan. 25, 2022 by \textbf{3:15pm (before class)}}
    \end{center}

    In this assignment, you will build a neural dependency parser using PyTorch. For a review of the fundamentals of PyTorch, please check out the PyTorch review session on Canvas. In Part 1, you will learn about two general neural network techniques (Adam Optimization and Dropout). In Part 2, you will implement and train a dependency parser using the techniques from Part 1, before analyzing a few erroneous dependency parses.
    \begin{questions}
        \titledquestion{Machine Learning \& Neural Networks}[8]
\begin{parts}

    \part[4] Adam Optimizer\newline
    Recall the standard Stochastic Gradient Descent update rule:
    \alns{
        \btheta &\gets \btheta - \alpha \nabla_{\btheta} J_{\text{minibatch}}(\btheta)
    }
    where $\btheta$ is a vector containing all of the model parameters, $J$ is the loss function, $\nabla_{\btheta} J_{\text{minibatch}}(\btheta)$ is the gradient of the loss function with respect to the parameters on a minibatch of data, and $\alpha$ is the learning rate.
    Adam Optimization\footnote{Kingma and Ba, 2015, \url{https://arxiv.org/pdf/1412.6980.pdf}} uses a more sophisticated update rule with two additional steps.\footnote{The actual Adam update uses a few additional tricks that are less important, but we won't worry about them here. If you want to learn more about it, you can take a look at: \url{http://cs231n.github.io/neural-networks-3/\#sgd}}

    \begin{subparts}
        \subpart[2] First, Adam adopts a commonly used technique called {\it momentum}, which keeps track of $\bm$, a rolling average of the gradients:
        \alns{
            \bm &\gets \beta_1\bm + (1 - \beta_1)\nabla_{\btheta} J_{\text{minibatch}}(\btheta) \\
            \btheta &\gets \btheta - \alpha \bm
        }
        where $\beta_1$ is a hyperparameter between 0 and 1 (often set to 0.9). Briefly explain in 2-4 sentences (you don't need to prove mathematically, just give an intuition) how using $\bm$ stops the updates from varying as much and why this low variance may be helpful to learning, overall.

        \ifans{
            It is obvious that only a small portion of $\bm$ may get affected in a single update, and therefore
            the {\it momentum} somehow reduces the amplitude of oscillations and avoids excessive change of $\btheta$ in a single step.
            This low variance helps maintain the efficiency of gradient descent, leading to faster convergence.
        }\newline

        \subpart[2] Adam extends the idea of {\it momentum} with the technique of {\it adaptive learning rates} by keeping track of  $\bv$, a rolling average of the magnitudes of the gradients:
        \alns{
            \bm &\gets \beta_1\bm + (1 - \beta_1)\nabla_{\btheta} J_{\text{minibatch}}(\btheta) \\
            \bv &\gets \beta_2\bv + (1 - \beta_2) (\nabla_{\btheta} J_{\text{minibatch}}(\btheta) \odot \nabla_{\btheta} J_{\text{minibatch}}(\btheta)) \\
            \btheta &\gets \btheta - \alpha \bm / \sqrt{\bv}
        }
        where $\odot$ and $/$ denote elementwise multiplication and division (so $\bz \odot \bz$ is elementwise squaring) and $\beta_2$ is a hyperparameter between 0 and 1 (often set to 0.99). Since Adam divides the update by $\sqrt{\bv}$, which of the model parameters will get larger updates? Why might this help with learning?

        \ifans{
            Model parameters with a scarce history of updates will get larger updates. This normalizes the update steps, avoids overshooting or a monotonically decreasing learning rate.
        }\newline

    \end{subparts}


    \part[4]
    Dropout\footnote{Srivastava et al., 2014, \url{https://www.cs.toronto.edu/~hinton/absps/JMLRdropout.pdf}} is a regularization technique. During training, dropout randomly sets units in the hidden layer $\bh$ to zero with probability $p_{\text{drop}}$ (dropping different units each minibatch), and then multiplies $\bh$ by a constant $\gamma$. We can write this as:
    \alns{
        \bh_{\text{drop}} = \gamma \bd \odot \bh
    }
    where $\bd \in \{0, 1\}^{D_h}$ ($D_h$ is the size of $\bh$)
    is a mask vector where each entry is 0 with probability $p_{\text{drop}}$ and 1 with probability $(1 - p_{\text{drop}})$. $\gamma$ is chosen such that the expected value of $\bh_{\text{drop}}$ is $\bh$:
    \alns{
        \mathbb{E}_{p_{\text{drop}}}[\bh_\text{drop}]_i = h_i \text{\phantom{aaaa}}
    }
    for all $i \in \{1,\dots,D_h\}$.
    \begin{subparts}
        \subpart[2]
        What must $\gamma$ equal in terms of $p_{\text{drop}}$? Briefly justify your answer or show your math derivation using the equations given above.\\
        \ifans{
            The probability of retaining a unit is $ (1-p_{\text{drop})} $. Therefore, the expected value of $ \bd \odot \bh $ is $(1-p_{\text{drop}})$.
            So $\gamma$ must equal to $ \frac{1}{1-p_{\text{drop}}} $.
        }\newline

        \subpart[2] Why should dropout be applied during training? Why should dropout \textbf{NOT} be applied during evaluation? (Hint: it may help to look at the paper linked above in the write-up.)
        \ifans{
            Applying dropout to a neural network during training somehow samples a ``thinned" network from it, and therefore
            prevents overfitting. However, there is no need to prevent overfitting during evaluation. Also, apply dropout during evaluation
            undermines the consistency of the model output.
        }\newline

    \end{subparts}
\end{parts}

        \newpage
        \titledquestion{Analyzing NMT Systems}[33]

\begin{parts}
    \part[3] In part 1, we modeled our NMT problem at a subword-level. That is, given a sentence in the source language, we looked up subword components from an embeddings matrix. Alternatively, we could have modeled the NMT problem at the word-level, by looking up whole words from the embeddings matrix. Why might it be important to model our Cherokee-to-English NMT problem at the subword-level vs. the whole word-level? (Hint: Cherokee is a polysynthetic language.) \\
    \ifans{
        Since Cherokee is a polysynthetic language, it may have a very large vocabulary and a large embedding matrix
        may be needed to implement effective word-level NMT models, which is impractical. Also, subword models may capture
        morphosyntactic relations for polysynthetic languages more effectively.
    } \fi \\

    \part[3] Transliteration is the representation of letters or words in the characters of another alphabet or script based on phonetic similarity. For example, the transliteration of {\cherokeefam Ꮳ⁠Ꮕ⁠Ꮤ⁠Ꮝ⁠Ꭺ} (which translates to ``do you know") from Cherokee letters to Latin script is tsanvtasgo. In the Cherokee language, ``ts-" is a common prefix in many words, but the Cherokee character {\cherokeefam Ꮳ} is ``tsa". Using this example, explain why when modeling our Cherokee-to-English NMT problem at the subword-level, training on transliterated Cherokee text may improve performance over training on original Cherokee characters.(Hint: A prefix is a morpheme.) \\
    \ifans{
        Training on transliterated Cherokee text could make it easier to separate words into morphemes, because the original
        Cherokee characters may contain inseparable combinations of morphemes.
    } \fi \\

    \part[3] One challenge of training successful NMT models is lack of language data, particularly for resource-scarce languages like Cherokee. One way of addressing this challenge is with multilingual training, where we train our NMT on multiple languages (including Cherokee). You can read more about multilingual training here:\newline \url{https://ai.googleblog.com/2019/10/exploring-massively-multilingual.html}.\newline How does multilingual training help in improving NMT performance with low-resource languages?\\
    \ifans{
        Multilingual model can somehow transfer parameters from relatively high-resource languages and improve NMT performance
        with low-resource languages.
    } \fi \\

    \part[6] Here we present a series of errors we found in the outputs of our NMT model (which is the same as the one you just trained). For each example of a reference (i.e., `gold') English translation, and NMT (i.e., `model') English translation, please:

    \begin{enumerate}
        \item Identify the error in the NMT translation.
        \item Provide possible reason(s) why the model may have made the error (either due to a specific linguistic construct or a specific model limitation).
        \item Describe one possible way we might alter the NMT system to fix the observed error. There are more than one possible fixes for an error. For example, it could be tweaking the size of the hidden layers or changing the attention mechanism.
    \end{enumerate}

    Below are the translations that you should analyze as described above. Only analyze the underlined error in each sentence. Rest assured that you don't need to know Cherokee to answer these questions. You just need to know English! If, however, you would like additional color on the source sentences, feel free to use a resource like \url{https://www.cherokeedictionary.net/} to look up words.

    \begin{subparts}
        \subpart[2]
        \textbf{Source Sentence:} \textit{{\cherokeefam ᏄᏩᏁᎰᎾ ᏕᎪᏣᎳᎩᏍᎬ, ᎯᎠ ᏄᏍᏕ ᏚᏏᎳᏛ: ᏧᏓᎴᏅᏓ ᏕᎪᏒᏍᎦ ᏧᏏᎳᏛᏙᏗ ᎠᏍᏓ ᎧᏅᏂᏍᎩ.        }}\newline
        \textbf{Reference Translation:} \textit{When \underline{she} was finished ripping things out, \underline{her} web looked something like this: }\newline
        \textbf{NMT Translation:} \textit{When \underline{it} was gone out of the web, \underline{he} said the web in the web.} \\
        \ifans{
            The NMT model uses ``it" and ``he" to refer to the same object. It seems that the NMT model has not really learned the
            target language grammar. To fix this, run more epoches, add more training data, and increase the size of the hidden layers may help.
        } \fi \\

        \subpart[2]
        \textbf{Source Translation}: \textit{{\cherokeefam ᎤᏍᏗ ᎢᏈᎬᎢ, ᎦᏙᏊᎢ? ᎤᏓᏛᏛᏁᎢ ᎤᏍᏗ ᎠᏧᏣ.}}\newline
        \textbf{Reference Translation}: \textit{What's wrong \underline{little} tree? the boy asked.}\newline
        \textbf{NMT Translation}: \textit{ The \underline{little little little little little} tree? asked him.} \\
        \ifans{
            The NMT Translation repeated the word ``little" for 5 times. It is possible that the model attends to the same parts
            of the source sentence and calculated a similar probability distribution. It may help to implement self-attention
            mechanism for the NMT model.
        } \fi \\

        \subpart[2]
        \textbf{Source Sentence:} \textit{{\cherokeefam “ᎤᏓᎸᏉᏗ ᏂᎨᏒᎾ,” ᎤᏛᏁ ᎰᎻ.}}\newline
        \textbf{Reference Translation:} \textit{\underline{“ ‘Humble,’ ”} said Mr. Zuckerman}\newline
        \textbf{NMT Translation:} \textit{\underline{“It’s not a lot,”} said Mr. Zuckerman.}\\
        \ifans{
            The model used ``it's not a lot" to express the meaning of ``humble." It is possible that the training data of the target
            language does not contain the word ``humble." To fix the problem, add more training data to the model or apply a character-based/hybrid NMT model
            in this situation.
        } \fi \\
    \end{subparts}

    \part[4] Now it is time to explore the outputs of the model that you have trained! The test-set translations your model produced in question \texttt{1-i} should be located in \texttt{outputs/test\_outputs.txt}.
    \begin{subparts}
        \subpart[2] Find a line where the predicted translation is correct for a long (4 or 5 word) sequence of words. Check the training target file (English); does the training file contain that string (almost) verbatim? If so or if not, what does this say about what the MT system learned to do? \\
        \ifans{
            Correct sequence of words: \textit{I want to see a little tree, and the frog}. \\
            The string \textit{I want to see} found verbatim in the training data. This indicates that the MT system somehow
            built a language model based on the target training data.

        } \fi \\

        \subpart[2] Find a line where the predicted translation starts off correct for a long (4 or 5 word) sequence of words, but then diverges (where the latter part of the sentence seems totally unrelated). What does this say about the model's decoding behavior? \\
        \ifans{\\
        \textit{It was tired of it, it was tired of the day, and in the day of the day he had tired of the morning.} \\
        It seems that the decoder is not aware of what it said a few time-steps ago.

        } \fi \\
    \end{subparts}

    \part[14] BLEU score is the most commonly used automatic evaluation metric for NMT systems. It is usually calculated across the entire test set, but here we will consider BLEU defined for a single example.\footnote{This definition of sentence-level BLEU score matches the \texttt{sentence\_bleu()} function in the \texttt{nltk} Python package. Note that the NLTK function is sensitive to capitalization. In this question, all text is lowercased, so capitalization is irrelevant. \\ \url{http://www.nltk.org/api/nltk.translate.html\#nltk.translate.bleu_score.sentence_bleu}
    }
    Suppose we have a source sentence $\bs$, a set of $k$ reference translations $\br_1,\dots,\br_k$, and a candidate translation $\bc$. To compute the BLEU score of $\bc$, we first compute the \textit{modified $n$-gram precision} $p_n$ of $\bc$, for each of $n=1,2,3,4$, where $n$ is the $n$ in \href{https://en.wikipedia.org/wiki/N-gram}{n-gram}:
    \begin{align}
        p_n = \frac{ \displaystyle \sum_{\text{ngram} \in \bc} \min \bigg( \max_{i=1,\dots,k} \text{Count}_{\br_i}(\text{ngram}), \enspace \text{Count}_{\bc}(\text{ngram}) \bigg) }{\displaystyle \sum_{\text{ngram}\in \bc} \text{Count}_{\bc}(\text{ngram})}
    \end{align}
    Here, for each of the $n$-grams that appear in the candidate translation $\bc$, we count the maximum number of times it appears in any one reference translation, capped by the number of times it appears in $\bc$ (this is the numerator). We divide this by the number of $n$-grams in $\bc$ (denominator). \newline

    Next, we compute the \textit{brevity penalty} BP. Let $len(c)$ be the length of $\bc$ and let $len(r)$ be the length of the reference translation that is closest to $len(c)$ (in the case of two equally-close reference translation lengths, choose $len(r)$ as the shorter one).
    \begin{align}
        BP =
        \begin{cases}
            1 & \text{if } len(c) \ge len(r) \\
            \exp \big( 1 - \frac{len(r)}{len(c)} \big) & \text{otherwise}
        \end{cases}
    \end{align}
    Lastly, the BLEU score for candidate $\bc$ with respect to $\br_1,\dots,\br_k$ is:
    \begin{align}
        BLEU = BP \times \exp \Big( \sum_{n=1}^4 \lambda_n \log p_n \Big)
    \end{align}
    where $\lambda_1,\lambda_2,\lambda_3,\lambda_4$ are weights that sum to 1. The $\log$ here is natural log.
    \newline
    \begin{subparts}
        \subpart[5] Please consider this example\footnote{Due to data availability, many Cherokee sentences with English reference translations are from the Bible. This example is John 1:5. The two reference translations are from the New International Version and the New King James Version translations of the Bible.}: \newline
        Source Sentence $\bs$: \textbf{{\cherokeefam ᎠᎴ ᎾᏍᎩ ᎢᎦ-ᎦᏘᏍᏗᏍᎩ ᎤᎵᏏᎬ ᏚᎸᏌᏕᎢ ᎤᎵᏏᎩᏃ ᎥᏝ ᏱᏚᏓᏂᎸᏤᎢ}}
        \newline
        Reference Translation $\br_1$: \textit{the light shines in the darkness and the darkness has not overcome it}
        \newline
        Reference Translation $\br_2$: \textit{and the light shines in the darkness and the darkness did not comprehend it}

        NMT Translation $\bc_1$: and the light shines in the darkness and the darkness can not comprehend

        NMT Translation $\bc_2$: the light shines the darkness has not in the darkness and the trials

        Please compute the BLEU scores for $\bc_1$ and $\bc_2$. Let $\lambda_i=0.5$ for $i\in\{1,2\}$ and $\lambda_i=0$ for $i\in\{3,4\}$ (\textbf{this means we ignore 3-grams and 4-grams}, i.e., don't compute $p_3$ or $p_4$). When computing BLEU scores, show your working (i.e., show your computed values for $p_1$, $p_2$, $len(c)$, $len(r)$ and $BP$). Note that the BLEU scores can be expressed between 0 and 1 or between 0 and 100. The code is using the 0 to 100 scale while in this question we are using the \textbf{0 to 1} scale.
        \newline

        Which of the two NMT translations is considered the better translation according to the BLEU Score? Do you agree that it is the better translation?\\
        \ifans{ \\
        For $\bc_1$, $p_1 = \frac{12}{13} $, $p_2 = \frac{10}{12}$, $len(c) = 13$, $len(r) = 13$, $BP=1$,\\ $BLEU = BP \times \exp(0.5\log{p_1}+0.5\log{p_2}) = 0.8770580193070293$.\\~\\
        For $\bc_2$, $p_1 = \frac{11}{13} $, $p_2 = \frac{9}{12}$, $len(c) = 13$, $len(r) = 13$, $BP=1$,\\ $BLEU = BP \times \exp(0.5\log{p_1}+0.5\log{p_2}) = 0.7966275068156915$.\\~\\
        According to the BLEU Score, $\bc_1$ is considered the better translation. I agree that it is the better translation.
        } \fi \\

        \subpart[5] Our hard drive was corrupted and we lost Reference Translation $\br_2$. Please recompute BLEU scores for $\bc_1$ and $\bc_2$, this time with respect to $\br_1$ only. Which of the two NMT translations now receives the higher BLEU score? Do you agree that it is the better translation?\\
        \ifans{\\
        For $\bc_1$, $p_1 = \frac{10}{13} $, $p_2 = \frac{8}{12}$, $len(c) = 13$, $len(r) = 13$, $BP=1$,\\ $BLEU = BP \times \exp(0.5\log{p_1}+0.5\log{p_2}) = 0.7161148740394329$.\\~\\
        For $\bc_2$, $p_1 = \frac{11}{13} $, $p_2 = \frac{9}{12}$, $len(c) = 13$, $len(r) = 13$, $BP=1$,\\ $BLEU = BP \times \exp(0.5\log{p_1}+0.5\log{p_2}) = 0.7966275068156915$.\\~\\
        According to the BLEU Score, $\bc_2$ is considered the better translation. I don't agree that it is the better translation because it is confusing and contains some syntactic errors.
        } \fi \\

        \subpart[2] Due to data availability, NMT systems are often evaluated with respect to only a single reference translation. Please explain (in a few sentences) why this may be problematic. In your explanation, discuss how the BLEU score metric assesses the quality of NMT translations when there are multiple reference transitions versus a single reference translation.\\
        \ifans{
            When a candidate translation is compared to a single reference translation, the representation of the relevant n-grams
            that should be matched to obtain a high BLEU Score is noisy.
            Therefore, the BLEU score for an appropriate translation may also be relatively low.
            When there are multiple reference translations, the NMT system gets more flexibility to interpret and decode the source
            sentence and is more likely to get a fair BLEU Score.
        } \fi \\
        \subpart[2] List two advantages and two disadvantages of BLEU, compared to human evaluation, as an evaluation metric for Machine Translation.\\
        \ifans{\\
        \textbf{Advantages}
            \begin{itemize}
                \item BLEU makes NMT evaluation fast, economical.
                \item It is language independent and easy to understand.
            \end{itemize}
            \textbf{Disadvantages}
            \begin{itemize}
                \item Its true quality remains far below human translations.
                \item It does not really consider semantics or sentence structures.
            \end{itemize}
        } \fi \\
    \end{subparts}
\end{parts}

    \end{questions}

    \Large{\textbf{Submission Instructions}}

\normalsize
You shall submit this assignment on GradeScope as two submissions -- one for ``Assignment 3 [coding]" and another for `Assignment 3 [written]":
\begin{enumerate}
    \item Run the \texttt{collect\_submission.sh} script to produce your \texttt{assignment3.zip} file.
    \item Upload your \texttt{assignment3.zip} file to GradeScope to ``Assignment 3 Coding".
    \item Upload your written solutions to GradeScope to ``Assignment 3 Written".
\end{enumerate}
\end{document}